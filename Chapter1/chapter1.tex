%!TEX root = ../thesis.tex
%*******************************************************************************
%*********************************** First Chapter *****************************
%*******************************************************************************

\chapter{Introduction}  %Title of the First Chapter

\ifpdf
    \graphicspath{{Chapter1/Figs/Raster/}{Chapter1/Figs/PDF/}{Chapter1/Figs/}}
\else
    \graphicspath{{Chapter1/Figs/Vector/}{Chapter1/Figs/}}
\fi


\section{What is global helioseismology?} %Section - 1.1 
Seismology is the study of vibration of material in a body on the surface to deduce its internal structure.


%\nomenclature[z-cif]{$CIF$}{Cauchy's Integral Formula}                                % first letter Z is for Acronyms 

\section{Frequency splittings in helioseismology} %Section - 1.2

The standard solar model assumed in this work is that of a spherically symmetric, non-rotating, non-magnetic, adiabatic, isotropic, and static (\snr). This \snr model admits wave like solutions in density perubations (and as a result, displacement, pressure, etc also) \cite{jcd_notes}. Normal modes are found to have a discrete spectrum of eigen-frequencies to the operator


The essential idea which drives this work is that one can systematically deduce internal structure parameters, such as convective flow, differential rotation, magnetic fields etc by analysing the frequency splittings of solar acoustic resonant modes. 

\subsection{Quasi Degenerate Perturbation Theory}

Quasi-degenerate pertubation theory requires us to account for mixing of modes which lie in a small neighbourhood on the frequency spectrum, and hence satisfy the quasi-degenerate condition.

\begin{equation}
|\omega_k^2 - \omref^2| < \tau^2
\end{equation}

%\section{Theoretical formulations} \label{Sec: Theory}

A general magnetohydrodynamic (MHD) description of the Sun using the MHD equations in its full vigour is rather cumbersome. Therefore, it is a standard practice (\cite{ritzwoller}\cite{lavely92}) to implement the degenerate perturbation theory in finding the changes in eigenfrequencies from the standard solar model. This model assume the sun to be spherically symmetric, non-rotating, non-magnetic and non-attenuating. The equation of motion (in temporal fourier space) for such a model is given by \citep{jcd_notes}

\begin{equation} \label{eqn:sol_wave_eqn}
\cL_0 \xiv = \rho \omega^2 \xiv  = - \grad (\rho c^2 \boldsymbol{\nabla} \cdot \boldsymbol{\xi} - \rho g \boldsymbol{\xi} \cdot \ev{r}) - g \ev{r} \boldsymbol{\nabla} \cdot (\rho \boldsymbol{\xi})
\end{equation}

where $\omega$ denotes the temporal frequency of oscillations, $c(r), g(r)$ and $\rho(r)$ are the radial functions of sounds speed, gravity and density. $\boldsymbol{\nabla}$ is the covariant spatial derivative operator. For all ensuing calculations and derivations we write equation \ref{eqn:sol_wave_eqn} in the form $\mathcal{L}_0 \boldsymbol{\xi} = \rho \omega^2 \boldsymbol{\xi}$ where the unperturbed wave operator $\mathcal{L}_0$ is self-adjoint (CITE SOMETHING?). Axis-symmetric or non-axissymmetric flows, rotations, asphericities, anisotropies and non-radial variations in $c, g, \rho$ can be captured as perturbation terms in equation \ref{eqn:sol_wave_eqn}. Although for the purpose of this study, we restrict ourselves to perturbations induced via presence of global scale magnetic fields only. 
\\

Because the solar eigenfunctions lack a toroidal component, we can denote the displacement field $\boldsymbol{\xi}(\boldsymbol{r},\omega)$ in the basis of spherical harmonics (and thereafter generalised spherical harmonics) as follows:

\begin{eqnarray} \label{eqn: xi_exp}
    \boldsymbol{\xi}(\boldsymbol{r}) &=& \sum_{k} U_{k}(r) Y_l^m(\theta,\phi) \ev{r}
 + V_{k}(r) \boldsymbol{\nabla}_1 Y_l^m(\theta,\phi) \\
     &=& \sum_{k} \xi_{k}^0(r) Y_{lm}^0(\theta,\phi) \ev{0} + \xi_{k}^-(r) Y_{lm}^-(\theta,\phi) \ev{-} + \xi_{k}^+(r) Y_{lm}^+(\theta,\phi) \ev{+}
 \end{eqnarray}
 
Here, $\boldsymbol{r} = (r,\theta,\phi)$ in spherical polar coordinate system with basis vectors $(\ev{r},\ev{\theta},\ev{\phi})$, $\grad_1 \equiv = \enc{\ev{\theta}\partial_{\theta} + \ev{\phi}\frac{1}{\sin\theta}\partial_{\phi} }$ and $k = (n,l,m)$ where $n$ is the radial order, $l$ is the angular degree and $m$ is the azimuthal order of the particular \snr mode.  These basis vectors are related to those in generalised spherical harmonics' basis as:
 
 \begin{equation}
     \begin{array}{ccc} \ev{-} = \frac{1}{\sqrt{2}}(\ev{\theta} - i \ev{\phi}), & \ev0 = \ev{r}, & \ev{+} = -\frac{1}{\sqrt{2}}(\ev{\theta} + i \ev{\phi}) \\ \end{array}
 \end{equation}
We work with normalised eigenfunctions $\boldsymbol{\xi}_k$ with the spherically symmetric background density $\rho(r)$ as the weight factor, which satisfy the orthonormality condition:

\begin{equation} \label{eqn: orthonormality}
  \inner{\xiv_{k'}}{\rho\xiv_{k}} = \delta_{n'n} \delta_{l' l} \delta_{m' m}  
\end{equation}
where the inner product \inner{}{} stands for $\inner{\Phi}{\Psi} \equiv \ints d^3\rvec \Phi^*(\rvec) \Psi(\rvec)$.

\section{Representation of Splitting Data}
Since each multiplet $\mode{n}{l}$ contains $2l+1$ singlet modes, it is tedious to give the degree of splitting in the mode by exact value of each split frequency $\omega_{nlm}$ by itself. Therefore, in helioseismology, splitting data is represented by numbers called splitting coefficients which describe the decomposition of $\delta\omega_{nlm} = \omega_{nlm}-\omega_{nl}$ in terms of some basis function over $m$ as follows
\begin{equation}
\omega_{nlm} = \omega_{nl} + \sum_{j=0}^{j_{max}} a^{nl}_j \cP_j(m)
\label{eq:a_def}
\end{equation}
where $\cP_{j}(m)$ with $j\in \{0,1,2,\ldots, j_{max}\}$ represents a $j_{max}+1$ dimensional orthogonal basis function of polynomials on the discrete space of $m$'s which run from $-l$ to $l$. Since, there are $2l+1$ points in this discrete domain, the (vector) space spanned by all functions on this domain is $(2l+1)$ dimensional, which in turn means $j_{max}$ cannot exceed $2l$. In practice, $a$-coefficients are recorded till a $j_{max}$ of $10$ (\cite{schou_data} for instace). We will use, as a standard, the basis functions prescribed in \cite{ritzwoller} which are Gram-Schmidt orthogonalised polynomials of increasing degree starting with $\cP_0(m) = l$. Given the normalisation condition mentioned above, and this starting condition, the polynomials become well defined. A recipe for obtaining these can be found in Appendix A of \cite{schou_pol_94}. Some properties to note about these polynomials are
\begin{itemize}
\item $\cP_j(m)$ is odd/even about $m=0$ if $j$ is odd/even respectively.
\item $\cP_j(m)$ has polynomial degree $j$ in $m$.
\item $\cP_j(m)$ contains only odd/even powers of $m$ if $j$ is odd/even respectively.
\item In limit $l \gg 1$, $\cP_j(m) \approx lP_{j}(m/l)$, where $P_j$ is Legendre polynomial of degree $j$.
\end{itemize}

Once we have the splitting data $\omega_{nlm}$, one can easily compute the $a$ coefficients multiplying both sides of Eq.(\ref{eq:a_def}) by $\cP_k(m)$,summing over all $m$, and finally using using the orthogonality condition $$\sum_{m=-l}^l \cP_{j}(m) \cP_{k}(m) = \delta_{jk} \sum_{m=-l}^l \enc{\cP_{j}(m)}^2$$
as

\begin{equation}
a^{nl}_j = \sum_{m=-l}^l \delta\omega_{nlm}\cP_j(m) \bigg/ \sum_{m=-l}^l \enc{\cP_j(m)}^2
\end{equation}


Note that, even though scaled Legendre polynomials $lP_j(m/l)$ are not perfectly orthogonal on a discretised domain (under the inner product $(A|B)\sum_{m=-l}^l A^*(m)B(m)$), they are still linearly independent. However their use should be aovisded as a basis to represent splitting data because value of $a$-coefficients will change slightly depending on $j_{max}$, i.e. how many $a$ coefficients are being fitted to the data; this occurs due to non-zero inner product between different basis functions. An orthogonal basis like $\cP_j$ solves this issue by making $a_j$ values independent of $j_{max}$.

Thus, henceforth whenever `$a$-coefficients' are referred to in this text, it shall be understood that the underlying basis functions are given by $\{\cP_j: j\in \{0,1,\dots,j_{max}\}\}$. The first few $\cP$ are given below

\begin{align}
\cP_{0}(m) &= l \\
\cP_1(m) &= m \\
\cP_2(m) &= \frac{3m^2-l(l+1)}{2l-1}
\end{align}