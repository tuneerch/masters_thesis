\chapter{Generalised Spherical Harmonics Formalism}\label{app_gsh}

\section{Formalism}
A general Formalism to describe complex tensor fields on the surface of a 2-sphere is that of Generalised Spherical Harmonics. (CREATORS) propose a generalisation of the spherical harmonic given by $Y_{lm}^N$, with an added index $N$, it reduces to the spherical harmonics when $N=0$.
\begin{equation}
Y_{lm}^0(\theta,\phi) = Y_l^m(\theta,\phi)
\end{equation}
These functions couple with the GSH unit vectors given by
\begin{equation}
\begin{array}{ccc} \ev{-} = \frac{1}{\sqrt{2}}(\ev{\theta} - i \ev{\phi}), & \ev0 = \ev{r}, & \ev{+} = -\frac{1}{\sqrt{2}}(\ev{\theta} + i \ev{\phi}) \\ \end{array}
\end{equation}
to form tensor spherical harmonics
\begin{equation}
\Yv_{lm}^{N} \equiv Y_{lm}^{N} \ev{\alpha_1}\ev{\alpha_2} \ldots \ev{\alpha_q}
\end{equation}
where $\alpha_1 + \alpha_2 +\ldots + \alpha_q = N$. These tensor functions by construction are eigenfunctions of the total angular momentum operator $\hat{\mathbf{J}} = \hat{\mathbf{L}} + \hat{\mathbf{s}}$, where the first part $\hat{\mathbf{L}}$ is the rotational generator for the scalar part of a field and the latter part $\hat{\mathbf{s}}$ generates rotation of the vector basis. Explicit expressions for $Y_{lm}^{N}$ can be found in \cite{DT98}. Eigenvalues are as below
\begin{equation}
\hat{J}_{z} \mathbf{Y}_{lm}^N = m\mathbf{Y}_{lm}^N
\end{equation}
\begin{equation}
\hat{J}^2 \mathbf{Y}_{lm}^N = l(l+1) \mathbf{Y}_{lm}^N
\end{equation}

\section{Conventions}
\begin{equation}
\gam{l} \equiv \frac{2l+1}{4\pi}
\end{equation}

\begin{equation}
\om{l}{N} \equiv \sqrt{\frac{(l+N)(l-N+1)}{2}}
\end{equation}
Note that $\om{l}{-N} = \om{l}{N+1}$.

\begin{equation}
g_{\mu\nu} \equiv \ev{\mu}\cdot \ev{\nu} = \left( \begin{matrix}
0 & 0 & -1 \\
0 & 1& 0  \\
-1 & 0 & 0
\end{matrix} \right)
\end{equation}

\begin{equation}
\ev{\mu}^* \cdot \ev{\nu} = \delta_{\mu\nu}
\end{equation}

\section{Spherical triple integral}
\begin{equation}
\int_0^{2\pi} d\phi \int_{0}^{\pi} d\theta\sin\theta Y_{l_1m_1}^{N_1}Y_{l_2m_2}^{N_2}Y_{l_3m_3}^{N_3} = 4\pi \gam{l_1}\gam{l_2}\gam{l_3} \wigfull{l_1}{l_2}{l_3}{N_1}{N_2}{N_3} \wigfull{l_1}{l_2}{l_3}{m_1}{m_2}{m_3}
\end{equation}
Explicit expressions for $\wigfull{l_1}{l_2}{l_3}{m_1}{m_2}{m_3}$ can be found in \cite{DT98}. The property $\enc{Y_{lm}^{N}}^* = (-1)^{m+N}Y_{l\bar{m}}^{\bar{N}}$ makes it useful while taking inner products of tensor fields in our analysis.
Note that works like \cite{lavely92}, \cite{hanasoge17} etc. use a slighlty different convention defining $Y_{lm}^{N}$ as $Y_{lm}^{N}/\gam{l}$ as in our convention. In this work, we have followed the convention of \cite{DT98}.
\section{Rotation}

Tensor GSH functions $\mathbf{Y}_{lm}^N$ obey the same rotation laws as spherical harmonics $Y_{l}^{m}$. Given two coordinate systems $(\theta,\phi)$ and $(\theta',\phi')$ such that $\phi=0$ and $\phi'=0$ planes coincide and on that plane $\theta' = \theta-\beta$, which corresponds to the primed axis being tilted by an angle $\beta$ to the unprimed axis, the GSH functions in the new coordinate system is given by
\begin{equation}
\mathbf{Y}_{lm}^{N}(\theta',\phi') = \sum_{m'=-l}^{l} d_{mm'}^{(l)}(\beta) \mathbf{Y}_{lm'}^{N}(\theta,\phi)
\end{equation}
where $d_{mm'}^{(l)}$ is a $(2l+1)\times (2l+1)$ real matrix. Explicit form of $d^{(l)}_{mm'}$ can be found in \cite{DT98} and a python subroutine for calculating this can be found in functions.py in the Github repository \cite{main_repo}.

